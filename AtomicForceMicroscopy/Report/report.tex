\documentclass[11pt,a4paper]{article}

\usepackage[english]{babel}
\usepackage[T1]{fontenc}
\usepackage[utf8]{inputenc}
\usepackage{graphicx}
\graphicspath{{../Figs/}}
\usepackage{float}
\usepackage{subcaption}
\usepackage[font=footnotesize,labelfont={sf,bf},textfont=sf,width=\textwidth]{caption}
\usepackage[margin=2cm]{geometry}
\usepackage[plainpages=false,pdfpagelabels,hypertexnames=false]{hyperref}
\usepackage[usenames,dvipsnames]{xcolor}
\usepackage{mathtools}
\usepackage[separate-uncertainty=true]{siunitx}
\usepackage{booktabs}


\title{\bfseries\textsc{Atomic Force Microscopy}}
\author{
Michele Masini\\ \small\texttt{\href{mailto:michele.masini@uni-ulm.de}{michele.masini@uni-ulm.de}}\and
Iyán Méndez Veiga\\ \small\texttt{\href{mailto:iyan.mendez-veiga@uni-ulm.de}{iyan.mendez-veiga@uni-ulm.de}}
}
\date{\today}


\begin{document}
\maketitle

\begin{abstract}
A compact Atomic Force Microscopy (AFM) from Nanosurf company (\href{https://www.nanosurf.com/en/products/naioafm-the-leading-compact-afm}{NaioAFM}) was used to perform measurements of the surface of three samples. Two different operation modes were used: static and tapping mode. Open source software \href{http://gwyddion.net/}{Gwyddion} was used to process the raw data.
\end{abstract}

\vspace{1.5cm}

\section{Introduction}

Since it was first developed in 1985 by \emph{Binning} et al., the Atomic Force Microscopy (AFM) \cite{Bhushan} has become a popular surface profiler for topographic and normal force measurements on the micro- to nanoscale. Not only AFMs, but also modified devices such as Lateral Force Microscopies (LFMs) or Friction Force Microscopies (FFMs), have found applications in different fields.

In this report, we will first describe the AFM technique, the two different modes that we tried (static and tapping or dynamic mode) and comment the issues we faced. And secondly, we will describe the processing of raw data obtained from the device using the open source software Gwyddion, as well as a general overview of AFM image artifacts, i.e., features which appear in the images that are not present in the original probed object.

\section{Materials and methods}

\subsection{AFM}

\subsection{NaioAFM}

\subsection{Gwyddion}


\section{Results and discussion}

\subsection{Static mode}

\subsubsection{Callibration}

\subsection{Tapping mode}

\subsubsection{Callibration}


\section{Conclusions}


\nocite{*}
\newpage
\bibliographystyle{unsrt}
\bibliography{references}


\end{document}
\documentclass[11pt,a4paper]{article}

\usepackage[english]{babel}
\usepackage[T1]{fontenc}
\usepackage[utf8]{inputenc}
\usepackage{graphicx}
\graphicspath{{../Figs/}}
\usepackage{float}
\usepackage{subcaption}
\usepackage[font=footnotesize,labelfont={sf,bf},textfont=sf,width=\textwidth]{caption}
\usepackage[margin=2cm]{geometry}
\usepackage[plainpages=false,pdfpagelabels,hypertexnames=false]{hyperref}
\usepackage[usenames,dvipsnames]{xcolor}
\usepackage{mathtools}
\usepackage[separate-uncertainty=true]{siunitx}
\usepackage{booktabs}
\usepackage[title]{appendix}

\title{\bfseries\textsc{Hall effect in semiconductors}}
\author{
Michele Masini\\ \small\texttt{\href{mailto:michele.masini@uni-ulm.de}{michele.masini@uni-ulm.de}}\and
Iyán Méndez Veiga\\ \small\texttt{\href{mailto:iyan.mendez-veiga@uni-ulm.de}{iyan.mendez-veiga@uni-ulm.de}}
}
\date{\today}


\begin{document}
\maketitle



\section{Introduction}
In this experiment, we will determine the carrier density ($p$) and the mobility ($\mu$) of a p-type semiconducting sample (Boron) as a function of the temperature, in the range $80K-300K$.

In order to obtain the carrier density we will use Hall-effect. To get the mobility we will measure the resistivity ($\rho$) by means of the van der Pauw method. 

Unfortunately, a semiconducting sample do not have a simple structure and the electric field depends on the position inside the material. This does not allow us to evaluate the Hall coefficient by means of a linear fit between current and voltage. We will show that considering a offset voltage, we will be able to evaluate our coefficient.

Intro about mobility bla bla

\section{Materials and methods}
Let us start explaining the evaluation of the carrier density.

Our sample is under the influence a constant magnetic field ($\vec{B}$). We can evaluate the drift velocity ($\vec{v}_d$) its charge carriers using the equations of motion of a particle in presence of an electric field:
\begin{equation*}
m\frac{d\vec{v}}{dt}+\frac{m}{\tau}\vec{v}_d=-e(\vec{E}+\vec{v}_d\times\vec{B})
\end{equation*} where $\tau$ is called relaxation time. The steady state solution of this equation is: 
\begin{equation}
v_d=-\frac{e\tau}{m}(\vec{E}+\vec{v}_d\times\vec{B})\equiv -\mu (\vec{E}+\vec{v}_d\times\vec{B})
\end{equation}Where $\mu$ is the \emph{mobility}. %In our case the electric field will be parallel to the surface of the sample and the magnetic field orthogonal, hence we can remove the vectors.

Using the definition of density of current in a p-doped matherial: $\vec{j}=ep\vec{v}_d$ (where $e$ is the elementar charge and $p$ is the holes density), we get the following expression for the electric field:
\begin{equation}
\vec{E}=\frac{1}{ep\mu}\vec{j}+\frac{1}{ep}\vec{j}\times\vec{B}\equiv \vec{E}_\parallel+\vec{E}_\perp
\end{equation}

Therefore, measuring the voltage between 2 random points P and N of the sample (as shown in figure {\tiny cite the figure}), what we are obtaining is:
\begin{equation}
U_{PN}=\int_P^N\vec{E}\cdot d\vec{s}=\int_P^{N'}\vec{E}_\perp\cdot d\vec{s}+\int_{N'}^N\vec{E}_\parallel\cdot d\vec{s}
\end{equation}
We will call the second term in the previous expression \emph{offset voltage} ($U_{off}$); this term can be measured in absence of magnetic field. The first term can be evaluated considering that the magnetic field is orthogonal to the sample surface:
\begin{equation}
\int_P^{N'}\vec{E}_\perp\cdot d\vec{s}=-\frac{B}{ep}\int_P^{N'}j\, ds=-\frac{B}{ep}\int_P^{N'}\frac{1}{d}\frac{dI}{ds}\, ds=-\frac{B}{epd}\int_P^{N'} dI=-\frac{BI}{epd}
\end{equation}
Defining $R_H=\frac{1}{ep}$ as the Hall coefficient, we get the relation that will allow us to measure the carrier density:
\begin{equation}
U_{PN}-U_{off}=-R_H\frac{BI}{d}
\end{equation}
\section{Results and discussion}


\section{Conclusions}

%\nocite{*}
%\vfill
%\bibliographystyle{unsrt}
%\bibliography{references}

\begin{appendices}


\end{appendices}
\end{document}
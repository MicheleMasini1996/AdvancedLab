\documentclass[11pt,a4paper]{article}

\usepackage[english]{babel}
\usepackage[T1]{fontenc}
\usepackage[utf8]{inputenc}
\usepackage{graphicx}
\graphicspath{{../Figs/}}
\usepackage{float}
\usepackage{subcaption}
\usepackage[font=footnotesize,labelfont={sf,bf},textfont=sf,width=\textwidth]{caption}
\usepackage[margin=2cm]{geometry}
\usepackage[plainpages=false,pdfpagelabels,hypertexnames=false]{hyperref}
\usepackage[usenames,dvipsnames]{xcolor}
\usepackage{mathtools}
\usepackage[separate-uncertainty=true]{siunitx}
\usepackage{booktabs}
\usepackage[title]{appendix}

\title{\bfseries\textsc{Hall effect in semiconductors}}
\author{
Michele Masini\\ \small\texttt{\href{mailto:michele.masini@uni-ulm.de}{michele.masini@uni-ulm.de}}\and
Iyán Méndez Veiga\\ \small\texttt{\href{mailto:iyan.mendez-veiga@uni-ulm.de}{iyan.mendez-veiga@uni-ulm.de}}
}
\date{\today}


\begin{document}
\maketitle



\section{Introduction}
In this experiment, we will determin the carrier density ($p$) and the mobility ($\mu$) of a p-type semicon-sample (Boron) as a function of the temperature, in the range $80K-300K$.

In order to obtain the carrier density we will use Hall-effect. To get the mobility we will measure the resistivity ($\rho$) by means of the van der Pauw method.


\section{Materials and methods}


\section{Results and discussion}


\section{Conclusions}

%\nocite{*}
%\vfill
%\bibliographystyle{unsrt}
%\bibliography{references}

\begin{appendices}


\end{appendices}
\end{document}